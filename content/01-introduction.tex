\chapter{Introduction}
A supply chain is a network of suppliers, factories, warehouses, distribution centers and retailers, through which raw materials are acquired, transformed, produced and delivered to the Customer[]. In the network we could find many entities whose could be considered as agents in the Multi-agent systems (MAS). Multi-agent System(MAS) are suitable…, there are many approaches are proposed in order to solve the problem in the supply chain mangement system. Such as negotiation-based Multi-agent System[].

The Supply Chain Management (\gls{scm}) world designed in \gls{scml} by \texttt{Yasser Mohammad} simulates a supply chain consisting of multiple factories that buy and sell products from one another. The factories are represented by autonomous agents that act as factory managers. Agents are given some target quantity to either buy or sell and they negotiate with other agents to secure the needed supplies or sales. Their goal is to turn a profit, and the agent with the highest profit (averaged over multiple simulations) wins \parencite{Mohammad2019}. It is characterized by profit-maximizing agents that inhabit a complex, dynamic, negotiation environment[]. There are two games built on the top of NegMAS which is the library for developing autonomous negotiation agents embedded in simulation environments.
..............

\subsection{Motivation}
Negotiation is a complex problem, in which the variety of settings and opponents that may be encountered prohibits the use of a single predefined negotiation strategy. Hence the agent should be able to learn such a strategy autonomously[]. The development of current machine learning algorithms and increased hardware resource make it possible, model the realistic environment to evaluate the problem with computer system. According to the modeled realistic environment, it will be easier to find more possible solutions.

In this work, we use some modeled negotiation environments, such as single agent environment(bilateral negotiation), and analyze whether deep reinforcement learning can be used to let agent learns some strategies autonomously in these environments. In contrast to single agent environment, in the supply chain environment, there are many agents with the same goal. After analyzing the simple environment, we need to explore whether multi-agent deep reinforcement learning can be used to obtain better results in multi-agent environment.

\textbf{How good strategy can be learned by deep reinforcement learning in single agent environment(bilateral negotiation)?}

\textbf{How good strategy can be learned by multi-agent deep reinforcement learning in multi-agent environment(concurrent negotiation)?}

\textbf{What is the difference between deep reinforcement learning strategies and other heuristic strategies?}

\subsection{Outline of this Work}
In the following, the other chapters of this work are listed and their content briefly presented.

\paragraph{Chapter 2: Background:}
This chapter contains basic knowledge and concepts that are necessary to understand the thesis. Firstly, some concepts from game theory are listed. These concepts are often discussed and used in autonomous negotiation. Secondly, utility function, some negotiation mechanisms are described in the section on autonomous negotiation. In addition, the basics and the historical development of artificial intelligence are presented. The focus of this chapter is on reinforcement learning.

\paragraph{Chapter 3: Related Work}
\paragraph{Chapter 4: Analyze}
\paragraph{Chapter 5: Methods and Experiments}
\paragraph{Chapter 6: Conclusions and Future Work}