\chapter{Einleitung}

Am Lehrstuhl für Interaktive Echtzeitsysteme (IES) des Karlsruhe Instituts für
Technologie (KIT) wurde von Philipp Wook die erste LaTeX-Vorlage zur Erstellung
von wissenschaftlichen Arbeiten erstellt. Diese basierte ihrerseits auf der
Vorlage von Matthias Pospiech von der Leibniz Universität Hannover. Die von
Matthias Pospiech und durch Philipp Wook stark erweitere Vorlage, hatte den
Anspruch die \enquote{eierlegende Wollmilchsau} zu sein und möglichst alle
Anwendungsfälle abzudecken.

Leider hatte diese Vorlage -- da sie ständig um zusätzliche \glspl{pkg} erweitert
wurde -- auch zwei meiner Meinung nach entscheidende Nachteile:
\begin{itemize}
  \item Es setzte auf den alten \glspl{bibtex}-Paketen auf anstatt des neueren
	  \texttt{biblatex}-Ökosystems. Dadurch war eine durchgängige
		\gls{utf8}-Unterstützung nicht möglich und die ein oder andere Konstellation
		von Umlauten hat immer mal wieder \enquote{geknallt}.
		
  \item Zum Erstellen von \index{Grafik|see{Bild}}\index{Bild!Vektor-}Vektorgrafiken mit einer hohen
	Druck- undTypografiequalität gibt es \gls{tikz}. Die alte Vorlage unterstütze
	zwar 	grundlegendes \gls{tikz}, aber bei vielen \gls{tikz}-Zusatzpaketen kam
	es zu 	Inkompatibilitäten mit anderen Paketen.
\end{itemize}

Diese Vorlage ist quasi ein Neudesign der Vorlage von Philipp Wook. Allerdings
hat sie intern nicht mehr allzu viele Gemeinsamkeiten. Im Wesentlichen wurde
versucht das Layout nachzuahmen, um eine halbwegs konsistentes Erscheinungsbild
zwischen Dokumenten mit alter und neuer Vorlage zu erreichen.

Natürlich ist so ein Neustart nicht ohne Nachteile möglich. Gegenwärtig ist
diese Vorlage noch sehr \enquote{schlank} und viele Möglichkeiten der alten
Vorlage sind noch nicht wieder nachgebaut. Dies wird in Zukunft und bei Bedarf
geschehen.

Einige Möglichkeiten der alten Vorlage werden aber nie wieder ihren Weg in diese
Vorlage finden. Dies betrifft alle LaTeX-Pakete die in irgendeiner Weise
\gls{postscript} benötigen. Um das volle Potential von \gls{utf8}, \gls{biblatex} und
\gls{tikz} ausnutzen zu können, ist die Verarbeitungskette auf
\begin{equation*}
\text{\texttt{LaTeX}-Quellcode} \xrightarrow{\text{\texttt{pdflatex}}} \text{\texttt{PDF}}
\end{equation*}
reduziert worden. Umwege wie
\begin{equation*}
\text{\texttt{LaTeX}-Quellcode} \xrightarrow{\text{\texttt{latex}}} \text{\texttt{DVI}} \xrightarrow{\text{\texttt{dvi2ps}}} \text{\texttt{PS}} \xrightarrow{\text{\texttt{ps2pdf}}} \text{\texttt{PDF}}
\end{equation*}
sind ausgeschlossen. Das bedeutet insbesondere, dass alle Optionen, die das
bekannte Paket \index{pstricks}\texttt{pstricks} bietet nicht mehr zur Verfügung
stehen. Allerdings bietet hier \gls{tikz} immer eine Ersatzlösung an. Das
einzige, was definitiv gar nicht mehr funktioniert und auch nicht durch
\gls{tikz} nachgestellt werden kann, ist das direkte Einbinden von
\gls{postscript}"~ bzw. \glstext{eps}-Abbildungen. Diese
müssen nun zunächst durch externe Tools in \glstext{pdf} konvertiert werden.

