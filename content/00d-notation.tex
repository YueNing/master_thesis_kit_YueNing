\addchap*{Notation}
\markboth{Notation}{}
Lower case letters are used for the values of random variables and for scalar functions. Capital ltters are used for random variables and major algorithms variables.
\addsec*{General identifier}

\begin{tabular}{p{1.5cm} p{12cm}}
$\alpha$, \dots, $\omega$ & Skalare \\
$\veca$, \dots, $\vecz$ & Skalar, Vektor, Funktionssymbol (oder Realisierung einer Zufallsvariablen) \\
$\rdveca$, \dots, $\rdvecz$ & Zufallsvariable (skalar bzw. vektoriell) \\
$\rdesveca$, \dots, $\rdesvecz$ & Schätzer für jeweilige Variable als Zufallsgröße \\
$\esveca$, \dots, $\esvecz$ & Realisierter Schätzer für jeweilige Variable \\
$\mA$, \dots, $\mZ$ & Matrix \\
$\rdmA$, \dots, $\rdmZ$ & Matrix als Zufallsgröße\\
$\sA$, \dots, $\sZ$ & Menge \\
$\ssA$, \dots, $\ssZ$ & Mengensystem \\
\end{tabular}

\addsec*{Special identifier}

\begin{tabular}{p{1.5cm} p{12cm}}
$s$ & State \\
$a$ & action \\
$\sS$ & set of nonterminal states \\
$\sA(s)$ & set of actions possible in state $s$ \\
$\sR$ & set of possible rewards \\

$t$ & discrete time step \\
$\mT$ & final time step of an episode \\
$\mS_t$ & state at $t$ \\
$\mA_t$ & action at $t$ \\
$\mR_t$ & reward at $t$ \\
$\mG_t$ & return (cumulative discounted reward) following $t$ \\
$\pi$ & policy, decision-making rule \\
$\pi(s)$ & action taken in state $s$ under \texttt{deterministic} policy $\pi$ \\
$\pi(a|s)$ & probability of taking action $a$ in state $s$ under stochastic policy $\pi$ \\
$p(s^',r|s, a)$ & probability of transitioning to state $s^'$, with reward $r$, from $s$, $a$ \\

$v_\pi(s)$ & value of state $s$ under policy $\pi$ (except return) \\
$v_*(s)$ & value of state $s$ under the optimal policy \\


\end{tabular}

\addsec*{General quantities}

\begin{tabular}{p{1.5cm} p{12cm}}
$\dsC$ & Menge der komplexen Zahlen \\
$\dsH$ & Poincaré Halbebene\\
$\dsN$ & Menge der natürliche Zahlen (ohne Null) \\
$\dsN_0$ & Menge der natürliche Zahlen mit Null \\
$\dsQ$ & Menge der rationalen Zahlen \\
$\dsQ^{>0}$, $\dsQ^{<0}$ & Menge der positiven bzw. negative rationalen Zahlen \\
$\dsR$ & Menge der reellen Zahlen \\
$\dsR^{>0}$, $\dsR^{<0}$ & Menge der positiven bzw. negative reellen Zahlen \\
$\dsZ$ & Menge der ganzen Zahlen \\
\end{tabular}

\addsec*{Special symbols}

\begin{tabular}{p{1.5cm} p{12cm}}
$\normDist( \mu, \sigma^2 )$ & Normalverteilung mit Erwartungswert $\mu$ und Varianz $\sigma$ \\
$\fisherDist_{r,s}$ & Fisher-Verteilung mir $r$ Zähler- und $s$ Nennerfreiheitsgraden\\
$\studentDist_s$ & Student-$t$-Verteilung mit $s$ Freiheitsgraden \\
$\diracDist_\xi$ & Ein-Punkt-Maß an der Stelle $\xi$ \\
$\chiSqDist_s$ & $\chiSqDist$-Verteilung mit $s$ Freiheitsgraden \\
\end{tabular}
