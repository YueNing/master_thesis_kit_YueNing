\chapter{Struktur der Vorlage}

Die Vorlage ist wie aus \cref{tab:files-dirs-of-template} zu entnehmen aufgebaut.
\begin{table}[htbp]
\centering
\begin{tabular}{l l l} \toprule
\bfseries Datei/Verzeichnis        & \bfseries Bedeutung  & \bfseries Benutzerinteraktion \\ \midrule
\verb#./thesis.tcp#                & \texttt{TeXnicCenter}-Projektdatei  &  indirekt \\
\verb#./main.tex#                  & \texttt{TeX}-Hauptdatei  &  nur \texttt{input}-Direktiven \\
\verb#./literature.bib#            & \texttt{BibLaTeX}-Quelldatei  &  ja \\
\verb#./figures-src/*#             & \texttt {TikZ}-Zeichnungen  &  ja \\
\verb#./images/*#                  & Binärformat-Bilder  &  ja \\
\verb#./content/*#                 & \texttt{LaTeX}-Kapitel  &  ja \\
\verb#./preamble.tex#              & Paketkonfigurationen  &  nein \\
\verb#./preamble-final-setup.tex#  & Paketkonfigurationen  &  nein \\
\verb#./preamble/*#                & Paketkonfigurationen  &  nein \\
\verb#./logos/*#                   & Institutslogos  &  nein \\
\verb#./figures-compiled/*#        & Temporäre Kompilate  &  nur löschen \\
\bottomrule
\end{tabular}
\caption{Dateien und Verzeichnisse der Vorlage}
\label{tab:files-dirs-of-template}
\end{table}

Die Datei \texttt{./thesis.tcp} ist ein \texttt{TeXnicCenter}-Projekt und sollte
in \texttt{TeXnicCenter} geladen werden.

Jedes \index{Kapitel}Kapitel der Arbeit wird im Verzeichnis \texttt{./content/}
als separate Datei gespeichert. Es empfiehlt sich als Dateiname das Schema
\texttt{nn-name.tex} zu verwenden, wobei \texttt{nn} die Nummer des Kapitels ist,
sodass die Dateien in der semantisch richtigen Reihenfolge sortiert angezeigt
werden. Per \texttt{input}-Direktive werden diese Dateien in der Datei \texttt{./main.tex}
eingebunden. Zudem kann das Titelblatt durch Verwendung von \texttt{content/00a-titlepage-hska}
und \texttt{content/00b-declaration-hska} auf das Layout der Hochschule angepasst werden.
Andere Veränderungen sollte man an der Datei \texttt{./main.tex} unterlassen.

\index{Bild}Bilder bzw. \index{Zeichnung|see{Bild}}Zeichnungen werden auf zwei
Arten eingebunden. Handelt es sich um Bilder, die mit einem externen Programm
erzeugt und in einem üblichen \index{Bild!Binär-}Binärformat abgespeichert
wurden (PNG, JPEG, TIFF, PDF, etc.) so werden diese im
Verzeichnis \texttt{./images/} abgelegt. Werden \index{Bild!Vektor-}(Vektor"~)Zeichnungen per
\gls{tikz} erzeugt, so wird die entsprechende Quelldatei im Verzeichnis
\texttt{./figures-src/} gespeichert. Werden des Kompilierens werden für jede
\gls{tikz}-Zeichnung im Verzeichnis \texttt{./figures-compiled/} mehrere
Dateien erzeugt. Der Inhalt der Verzeichnisses kann gefahrlos gelöscht werden.
\textbf{Bitte beachten: Das Verzeichnis \texttt{./figures-compiled/} muss einmalig manuell erzeugt werden.}

